\documentclass{article}
\usepackage{tikz}
\usepackage{float}
\usetikzlibrary{arrows,positioning,shapes.geometric}
\setlength{\headsep}{5pt}

\title{%
Redes Neuronales \\
Análisis del modelo Integrate-and-Fire \\*[23pt]
Trabajo Práctico 2 \\
}
\date{2020}
\author{Igor Andruskiewitsch}

\begin{document}
    \maketitle

\section{Introducción}

\subsection{Modelo Integrate-and-Fire}

Este trabajo está orientado a comprender el modelo {\bf Integrate-and-Fire }, que modela la evolución temporal del potencial de membrana $ V_m(t) $ al tiempo $ t $, entre el interior y el exterior de una neurona genérica. Este modelo es descrito como la siguiente ecuación diferencial ordinaria (ODE):

\[ \dot{V_m}(t) = {1 \over \tau_m} (E_L - V_m(t) + R_m I_e(t)) \]

Donde:

\begin{itemize}
    \item {$ E_L $ es el potencial en reposo (mV) }
    \item {$ I_e(t) $ es la corriente eléctrica externa inyectada en el tiempo $ t $ (mV) }
    \item {$ R_m $ es la resistencia en megaOhms ($ M\Omega $) }
    \item {$ \tau_m $ es el tiempo característico de la membrana }
\end{itemize}

El desafío de este modelo recae en la corriente externa $ I_e(t) $, como esta función es desconocida, no podemos encontrar una solución analítica al problema. Aún así, podemos utilizar distintos métodos para aproximar el comportamiento de la neurona, que vamos a ver más adelante.

\pagebreak

\section{Resolución Analítica}

Si consideramos la corriente externa como una constante $ I_e(t) = I_e $, podemos buscar una solución analítica a nuestro modelo. Además, vamos a graficar la solución para $ 0ms \leq t \leq 200ms $ con los siguientes valores para los parámetros:

\[ V_m(t = 0) = E_L = -65mV, \qquad R = 10 M \Omega, \qquad V_{th} = -50 mV, \qquad \tau_m = 10ms \]

Cabe aclarar que el valor de $ V_{th} $ no se va a considerar para este primer paso, ya que este valor marca el umbral para el cual nuestro $ I_e(t) $ afecta el potencial de la neurona. Como en este caso $ I_e(t) $ es constante, no tendremos que considerar este valor.

Comenzando con la resolución, teniendo en cuenta que ahora $ I_e $ es una constante, podemos reescribir la ecuación de la siguiente forma:

\[ \dot{V_m}(t) = A V_m(t) + B \]

Donde:

\[ A = { {- 1} \over \tau_m } \qquad B = { {R_m I_e + E_L} \over \tau_m } \]


Ahora comenzamos a resolver esta ecuación:

\[ V'(t) = A V(t) + B  \Rightarrow { V'(t) \over {A V(t) + B} } = 1 \]

Hacemos el reemplazo $ U(t) = A V(t) + B $, $ U'(t) = A V'(t) $, luego:

\[ V'(t) = { U'(t) \over A } \Rightarrow { U'(t) \over A U(t) } = 1 \Rightarrow { U'(t) \over U(t) } = A \Rightarrow { \int {U'(t) \over U(t)} } = A t + C \]

Ahora, sabiendo que $ \int { {f'(x) \over f(x)} dx } = \ln (f(x))  $, vemos que:

\[ \ln (U(t)) = A t + C  \Rightarrow U(t) = e^{A t + C} = A V(t) + B \]
\[ \Rightarrow V(t) = {1 \over A} ( e^{At + C} - B ) = ({ e^C \over A }e^{At}) - { B \over A } \]

Tomando $ c^1 = { e^C \over A } $ y reemplazando $ A $ y $ B $ obtenemos:

\[ c^1 e^{-t \over \tau_m } - { {( R_m I_e + E_L ) \over \tau_m } \over {-1 \over \tau_m } } = c^1 e^{-t \over \tau_m } + R_m I_e + E_L \]

Ahora necesitamos conocer el valor de $ c^1 $. Esto lo podemos hacer ya que conocemos el valor inicial en $ t = 0 $:

\[ c^1 e^{ -t \over \tau_m } + R_m I_e + E_L = -65 \Rightarrow c^1 - 65 = R_m I_e - 65 \Rightarrow c^1 = R_m I_e \]

Reemplazando, obtenemos:

\[ V_m(t) = (R_m I_e) e^{-t \over \tau_m} + R_m I_e + E_L = (R_m I_e) (e^{-t \over \tau_m} + 1) + E_L \]

\end{document}
